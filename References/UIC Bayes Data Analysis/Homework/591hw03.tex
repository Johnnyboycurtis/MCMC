

\documentclass[a4paper,12pt]{article}
\usepackage{amsmath, amsthm, amssymb, mathrsfs}
\usepackage{fullpage}
\usepackage{verbatim, graphicx, multirow, url}
\usepackage[usenames]{color}
\usepackage{ifthen}

%\parindent 0pt

%\renewcommand{\figurename}{Fig}

\newcommand{\E}{\mathsf{E}}
\newcommand{\var}{\mathsf{V}}
\newcommand{\cov}{\mathsf{C}}
\newcommand{\prob}{\mathsf{P}}

\newcommand{\eps}{\varepsilon}
\renewcommand{\phi}{\varphi}

\newcommand{\iid}{\overset{\text{\tiny iid}}{\sim}}
\newcommand{\cd}{\overset{\text{\sc d}}{\to}}
\newcommand{\cp}{\overset{\text{\sc p}}{\to}}

\newcommand{\nm}{\mathsf{N}}
\newcommand{\ber}{\mathsf{Ber}}
\newcommand{\bin}{\mathsf{Bin}}
\newcommand{\pois}{\mathsf{Pois}}
\newcommand{\expo}{\mathsf{Exp}}
\newcommand{\gam}{\mathsf{Gam}}
\newcommand{\unif}{\mathsf{Unif}}
\newcommand{\be}{\mathsf{Beta}}

\newcommand{\Xbar}{\bar X}%{\overline{X}}
\newcommand{\xbar}{\bar x}%{\overline{x}}

\newcommand{\grad}{\nabla}



\begin{document}

\noindent \large \textbf{Stat 591 -- Homework 03} \hfill \textbf{Due: Wednesday 10/16} \normalsize

\medskip

\noindent \emph{Your group should submit a write-up that includes solutions the problems stated below, along with any \underline{relevant} pictures/graphs or computer code/output.}  

\medskip

\begin{enumerate}

\item Let $(X_1,\ldots,X_n) \mid \theta \iid \nm(\theta, 1)$ and consider an improper flat prior for $\theta$, i.e., the prior density is $\pi(\theta) = 1$.  
\begin{enumerate}
\item If $X_1,\ldots,X_n \iid \nm(\theta^\star, 1)$, show that the posterior is consistent.  
\item Design a simulation study that demonstrates the posterior consistency result you derived.  Explain the results of your simulation.  

\emph{Hints:}
\begin{itemize}
\item Fix $\theta^\star=0$ and $\eps=0.01$, say, and define 
\[ Q_n = \Pi(\theta: |\theta-\theta^\star| > \eps \mid X_1,\ldots,X_n), \]
a random variable as a function of $X_1,\ldots,X_n \iid \nm(\theta^\star, 1)$. 
\item For a given $n$, get a sample of size $M$ from the distribution of $Q_n$ by simulating $M$ sequences $X_1,\ldots,X_n \iid \nm(\theta^\star, 1)$ and computing the corresponding $Q_n$ for each; take $M=1000$, say.  
\item Repeat this experiment for several values of $n$, ranging from relatively small to relatively large.  
\item Summarize the results by plotting, say, the 50th, 75th, and 95th percentiles of the $Q_n$ distribution as a function of $n$.  
\end{itemize}
\end{enumerate}

\item Problem 4.2 in [GDS], page 119.

\emph{Hint:} Let $\theta^\star=\theta_i$.  Because the problem is discrete, it suffices to show that 
\[ \Pi(\theta=\theta_i \mid X_1,\ldots,X_n) \to 1 \quad \text{in $\prob_{\theta_i}$-probability}. \]
Look at the hint given in [GDS].  If $r=i$, then $Z_r \equiv 0$; if $r \neq i$, then $Z_r$ converges in $\prob_{\theta_i}$-probability to $-K(\theta_i, \theta_r)$, where $K(\theta_i,\theta_r) = \int \log\{ f_{\theta_i}(x) / f_{\theta_r}(x)\} f_{\theta_i}(x) \,dx$ is the \emph{Kullback--Leibler divergence} of $f_{\theta_r}(x)$ from $f_{\theta_i}(x)$.  By Jensen's inequality, we know that $K(\theta_i, \theta_r)$ is strictly positive for $r \neq i$.  

\item Problem 4.3 in [GDS], page 119.

\emph{Hint:} For given $\eps > 0$, define a sequence of random variables 
\[ Q_n(X_1,\ldots,X_n) = \Pi(|\theta-\theta^\star| > \eps \mid X_1,\ldots,X_n) \]
and a sequence of subsets of the sample space 
\[ A_n = \{(x_1,x_2,\ldots): |\hat\theta_n(x_1,\ldots,x_n)-\theta^\star| \leq \eps/2\}. \]
Dropping the dependence on $(X_1,\ldots,X_n)$ in the notation, write 
\[ Q_n = Q_n \cdot I_{A_n} + Q_n \cdot I_{A_n^c}, \]
where, e.g., $I_{A_n} = I_{A_n}(X_1,\ldots,X_n)$ is the indicator of $A_n$.  Now show that both terms in this sum go to zero as $n \to \infty$ in $\prob_{\theta^\star}$-probability.  

\item Problem 4.4 in [GDS], page 119.

\emph{Hint:} Note that [GDS] defines $L_n(\theta)$ to be the log-likelihood function, what I would usually write as $\ell_n(\theta)$.  Then 
\[ \frac{L_n(\theta) - L_n(\theta^\star)}{n} = (\Xbar-\theta^\star) (\theta-\theta^\star) - \frac{(\theta-\theta^\star)^2}{2}. \]
For a fixed real number $a$, define the function $g_a(z)=az-z^2/2$.  Argue that this function is negative for $|z| > 2|a|$.  Now note that the right-hand side of the above display is $g_a(z)$ with $a=\Xbar-\theta^\star$ and $z=\theta-\theta^\star$.  You'll need the law of large numbers to control $\Xbar-\theta^\star$.  

\item Recall Problem~\#4 in Homework~2, the one on multinomial with a Dirichlet prior and the agreement application.  Argue that the posterior distribution of $n^{1/2}(\kappa-\hat\kappa_n)$ is asymptotically normal,\footnote{In this problem, the parameter $\theta$ is four-dimensional but, as I said in class, all the posterior convergence theorems hold, with obvious adjustments, for any finite-dimensional problem.} and write down an expression for the variance in the normal approximation.  Compare this to the results for the asymptotic distribution of the MLE $\hat\kappa_n$ you found in Homework~1.   

\end{enumerate}




\end{document}
